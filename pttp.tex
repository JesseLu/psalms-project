\documentclass[a5paper]{book}
\usepackage[a5paper]{geometry}

\title{Praying through the Psalms}
\author{Jesse Lu}

\newcommand{\q}{\textit}
\newcommand{\Q}{\quote}
\begin{document}
\maketitle
% \tableofcontents
\chapter{Psalm 1}
Psalm 1 is the first psalm in the book of Psalms
    because it forms the foundation of every believer's life.
Why is this psalm so critical to the Christian life?
Because it identifies what must be the distinguishing characteristic
    of every Christian---a delight for God's word (v. 3).

\section{The separation of the righteous and the ungodly}
The first truth we must understand from Psalm 1
    is that there are only two kinds of people on earth:
    the righteous and the ungodly.
Every human being is either a righteous or an ungodly person.
Every human is either in the category of the righteous or 
    else must be in the category of the ungodly;
    no one can be in both categories at the same time,
    and there is no third category that anyone can belong to.

\subsection{The general identification of the ungodly}
We know this to be true, first,
    because of the very general description of the ungodly in Psalm 1.
The ungodly are very simply described in verse 4 in this way, 
    \q{The ungodly are not so};
    which simply means that the ungodly man is not like the righteous man
    (who is described in verses 1-3)!

This means that the psalmist is not referring to some kind of extreme sinner
    when he uses the terms \emph{ungodly}, \emph{sinner}, or \emph{scoffer}.
Instead, the psalmist is simply referring to \emph{those who are not righteous!}
An equally valid translation for the word \q{ungodly}
    would be the word \q{unrighteous}.

This is why every human belongs to one and only one of these two groups
    and not to any third or fourth group---because 
    every human must either be righteous or else, must not.

\subsection{The divine separation of the righteous and the ungodly}
The second reason why we can know that
    everyone is either righteous or ungodly is
    because these two groups are divinely separated by God's judgment.
Verse 5 reads,
    \Q{Therefore the ungodly shall not stand in the judgement, \\
    Nor sinners in the congregation of the righteous.}

A judgment results in only two possible outcomes:
    you either pass the judgment, or you don't.
Why is it impossible 
Since verse 5 says that the ungodly will not pass
\section{The distinguishing characteristic of the righteous}
\section{The blessing of the righteous}
\end{document}
