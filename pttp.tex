\documentclass[a5paper]{book}
\usepackage[a5paper]{geometry}

\title{Praying through the Psalms}
\author{Jesse Lu}

\newcommand{\q}[1]{\textit{#1}}
\newcommand{\Q}[1]{\begin{quote}#1\end{quote}}
\begin{document}
\maketitle
\tableofcontents

\chapter{Psalm 1}
Psalm 1 is the first psalm in the book of Psalms
    because it forms the foundation of every believer's life.
Why is this psalm so critical to the Christian life?
Because it identifies what must be the distinguishing characteristic
    of every Christian---a delight for God's word (v. 2).

\section{The separation of the righteous and the ungodly}
The first truth we must understand from Psalm 1
    is that there are only two kinds of people on earth:
    the righteous and the ungodly.
Every human being is either a righteous or an ungodly person.
Every human is either in the category of the righteous or 
    else must be in the category of the ungodly;
    no one can be in both categories at the same time,
    and there is no third category that anyone can belong to.

\subsection{The general identification of the ungodly}
We know this to be true, first,
    because of the very general description of the ungodly in Psalm 1.
The ungodly are very simply described in verse 4 in this way, 
    \q{The ungodly are not so};
    which simply means that the ungodly man is not like the righteous man
    (who is described in verses 1-3)!

This means that the psalmist is not referring to some kind of extreme sinner
    when he uses the terms \emph{ungodly}, \emph{sinner}, or \emph{scoffer}.
Instead, the psalmist is simply referring to \emph{those who are not righteous!}
An equally valid translation for the word \q{ungodly}
    would be the word \q{unrighteous}.

This is why every human belongs to one and only one of these two groups
    and not to any third or fourth group---because 
    every human must either be righteous or else, must not.

\subsection{The divine separation of the ungodly}
The second reason why we can know that
    everyone is either righteous or ungodly is
    because these two groups are divinely separated by God's judgment.
Verse 5 reads,
    \Q{Therefore the ungodly shall not stand in the judgement, \\
    Nor sinners in the congregation of the righteous.}

A judgment results in only two possible outcomes:
    you either pass the judgment, or you don't.
Why is it impossible to be in both groups at the same time?
Because it is impossible to both pass and fail a judgement.
Instead, we will either stand in the judgement of God,
    or else we will not (verse 5a).

The second half of verse 5 further strengthens the notion
    of humanity divided into two distinct groups,
    because it refers to the congregation (or grouping) of the righteous
    which sinners are not a part of.
Therefore, everyone is either one of the righteous,
    who have a place in this congregation of the righteous,
    or a sinner, who does not.
    
\section{The distinguishing characteristic of the righteous}
The fact that men are only divided into either 
    the righteous or else the ungodly 
    then begs the questions,
    ``Who are the righteous?'' and
    ``How can I be part of the righteous?''
These questions are answered in verse 1 and 2,
    which expose the distinguishing characteristic of the righteous.
    \Q{Blessed is the man \\
    Who walks not in the counsel of the ungodly, \\
    Nor stands in the path of sinners, \\
    Now sits in the seat of the scornful; \\
    But his delight is in the law of the Lord, \\
    And in His law he meditates day and night.}
    
The distinguishing characteristic of the righteous man,
    in the negative sense,
    is that his life is antithetical to that of the ungodly.
Verse 1 says that he does not walk in their counsels, 
    stand in their paths, or sit in their seats.
This is not to say that the righteous is isolated and
    has no interaction with the ungodly;
    rather, it means that when the righteous
    come in contact with the ungodly,
    the ungodly man is meeting someone who's life is heading
    in a completely opposite direction from his own.

This contrast becomes clear as the central distinguishing characteristic
    of the righteous is presented in verse 2,
    ``But his delight is in the law of the Lord,''.

How is the righteous man identified? 
And how can we test if we are righteous men or women ourselves?
Psalm 1 gives us a clear test:
    Do we delight in the Scripture?

The test is \emph{not} whether we know the Bible,
    or simply read the Bible,
    or go to church to hear the Bible preached.
Psalm 1 does not even identify the righteous as the one
    who \emph{has delighted} in God's Word.
No, the righteous man of God is known by this primary characteristic:
    a joyous love and delighting in the word of God.

What is the root cause of trouble in the life of the believer?
And what is origin of our sin and unfaithfulness to our Lord?
Psalm 1 informs us that our root problem is our delighting in other things,
    over and above God's word.

And not only does Psalm 1 identify the root origin of our sin,
    but it also gives us the remedy for our situation:
    to delight in God's word.

The righteous man's delight in God's word is the overpowering characteristic
    of his life.
Without exaggeration, we may say that he is obsessed with Scripture.
Verse 2 ends, ``And in His law he meditates day and night.''
The faithful Christian life is characterized by
    an all-consuming love and joy in the word of God.
To what extent?
To the extent that it is his continual meditation throughout each day.
    
\section{The blessing of the righteous}
Having seen the \emph{separation} of the righteous,
    and the \emph{distinguishing characteristic} of the righteous,
    we now turn our attention to the \emph{blessing} of the righteous.
This is, in fact, the main point of Psalm 1 which begins with, 
    ``Blessed is the man\ldots''.
Verse 3 describes the blessing of the one who delights in the law of the Lord 
    in this way,
    \Q{He shall be like a tree \\
    Planted by the rivers of water, \\
    That brings forth its fruit in its season, \\
    Whose leaf also shall not wither, \\
    And whatever he does shall prosper.}

This verse describes the blessed life of the righteous man by way of analogy,
    ``He shall be like a tree'',
    and what a remarkably blessed tree the righteous man is!

First, we see that this tree is abundantly supplied
    in that it is, ``Planted by the rivers of water,''.
This means that the tree does not need to depend on the weather (i.e. rainfall)
    for its nourishment;
    instead, it is constantly nourished,
    not just by a river,
    but even a multitude of rivers!
What is the blessing of the righteous man?
The righteous man's blessing is to be abundantly and continually supplied 
    and provided by God in every situation.

Secondly, Psalm 1:3 speaks of the fruitfulness of this continually nourished tree,
    ``That brings forth its fruit in its season,''.
Not only is such a man fruitful, or spiritually productive,
    but his life has this characteristic in a regular, continuous fashion.
The culture may change, his circumstances may ebb and flow,
    but just as he is continually supplied by God in whatever circumstance,
    so he is fruitful both in times of plenty and in times of drought.

The parallels in Scripture are many;
    for instance, this fruitful tree is also
    the fruitful seed sown on the good soil in Matthew 13:23,
    ``who indeed bears fruit and produces:
    some a hundredfold, some sixty, some thirty.''
Jesus' words from John 15 also describe a similar man,
    this time as a branch who is connected to the vine that is Jesus Christ Himself.
This man, abiding in Jesus Christ, is not only perfectly provided for, 
    but is fruitful as well, 
    ``He who abides in Me, and I in him, 
    bears much fruit; for without Me you can do nothing.'' (John 15:5).

In the analogy of the righteous man as a tree, 
    Psalm 1:3 shows that he is not only 
    abundantly nourished and continually fruitful,
    but that he is divinely protected,
    ``Whose leaf also shall not wither;''.
This is the wise man of who hears and does the words of Christ
    and, ``the rain descended, the floods came, and the winds blew and beat 
    on that house; and it did not fall, for it was founded on the rock.''
    (Matthew 7:25).

There is a divine invincibility for the one whose delight is in
    the law of the Lord as personified by Paul who testified,
    ``We are hard-pressed on every side, yet not crushed;
    we are perplexed, but not indespair;
    persecuted, but not forsaken;
    struck-down, but not destroyed--'' (2 Corinthians 4:8-9).
And from this we understand the spiritual intent of the verse,
    that unwithering leaves translate not into a life devoid of hardship,
    but a faith that cannot be broken,
    ``who are kept by the power of God through faith for salvation ready
    to be revealed in the last time.'' (1 Peter 1:5).

The last line of verse 3 is the blanket statement
    which rises above the tree analogy,
    ``And whatever he does shall prosper.''
What is the way of properity, true spiritual propserity,
    for the Christian?
It is simply this, an all-consuming desire for the word of God.

\section{The blessor of the righteous}
The blessedness of the righteous is not simply a result of his character;
    instead, the properity of the righteous has one ultimate cause, and one alone,
    God.
This is what is communicated to us in the last verse of Psalm 1,
    \Q{For the Lord knows the way of the righteous, \\
    But the way of the ungodly shall perish.}

Why is the life of the righteous flourish like the tree
    described in verse 3?
Because it is the Lord who is tending to the tree.
That He ``knows the way of the righteous'' is not primarily
    referring to His omniscience, which is true of all things,
    but is referring to His special care and attention 
    with which He cares for the righteous.
He knows the righteous' way in that He has set His face upon them for good;
    the light of His countenance shines upon the righteous.

The righteous tree is watered by the rivers which God supplies,
    and is fruitful because it is pruned by God,
    and is invicible because of God's protection.
He is the source of properity for the righteous man.  
The prosperity of the righteous originates from God,
    who response to the one who gives full attention to the word
    is to shower him with abundant grace.

It is the Christian's greatest delight, then,
    to be known by God in this special manner;
    especially since the alternative is to be the ungodly
    whose way shall perish.
Why does the way of the ungodly perish?
Simply because having denied and rejected their Creator,
    they have been denied and rejected by Him.
And so their life is devoid of His goodness,
    ``The ungodly are not so,
    but are like the chaff which the wind drives away.''
    (Psalm 1:4).

%% Prayer.

\end{document}
