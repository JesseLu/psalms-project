\chapter{Psalm 2}
Psalms 1 and 2 form the twin foundations of the rest of the book of Psalms.
We conclude this because of the many parallels between these first two psalms.
Both feature a blessing (Psalm 1:1 and Psalm 2:12) and 
    both have as their focus the Word of God,
    whether it is the written Word in the first Psalm
    or the incarnate Word in the second.

For all their parallels, there exist many differences between these psalms as well.
The primary difference is that while Psalm 1 focuses on the righteous,
    Psalm 2's main focus is the wicked.
In fact, just as the purpose of Psalm 1 is to exhort us to be like the righteous,
    the purpose of Psalm 2 is complementary---namely
    to \emph{not} be like the wicked.

Thus these two psalms serve the same purpose, 
    to serve God and forsake sin,
    in complementary ways;
    first by showing us the good of delighting in His Word
    and then by showing us the evil in rejecting it.
An understanding of psalms 1 and 2 in this way fulfills
    the exhortation of the apostle Paul,
    ``Therefore consider the goodness and severity of God;''
    (Romans 11:22).

\section{The rejection of the wicked}
% Who they are: nationa, people, kings, rulers.
% Who they are opposed to: the Lords and His Annointed.
% They're beef against them (restrictions).
\section{The response of the Father}
% contempt
% wrath
% Establishing His Annointed.
\section{The reign of the Son}
% His person,
% His reign
% His victory
\section{The reasoning of the Psalmist}
