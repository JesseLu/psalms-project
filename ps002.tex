\chapter{Psalm 2}
Psalms 1 and 2 form the twin foundations of the rest of the book of Psalms.
We conclude this because of the many parallels between these first two psalms.
Both feature a blessing (Psalm 1:1 and Psalm 2:12) and 
    both have as their focus the Word of God,
    whether it is the written Word in the first Psalm
    or the incarnate Word in the second.

For all their parallels, there exist many differences between these psalms as well.
The primary difference is that while Psalm 1 focuses on the righteous,
    Psalm 2's main focus is the wicked.
In fact, just as the purpose of Psalm 1 is to exhort us to be like the righteous,
    the purpose of Psalm 2 is complementary---namely
    to persuade us \emph{not} be like the wicked.

Thus these two psalms serve the same purpose, 
    to serve God and forsake sin,
    in complementary ways;
    first by showing us the good of delighting in His Word
    and then by showing us the evil in rejecting it.
An understanding of psalms 1 and 2 in this way fulfills
    the exhortation of the apostle Paul,
    ``Therefore consider the goodness and severity of God;''
    (Romans 11:22).

\section{The rejection of the wicked}
% Who they are: nations, people, kings, rulers.
Just as Psalm 1 divided all of humanity into two groups,
    the righteous and the wicked, so it is in Psalm 2.
Once again, the category of the wicked is exceedingly broad,
    including nations, people, kings, and rulers (v. 1);
    that is, the people of the world from the most common to the most exalted.
The righteous, once again, are a defined very specifically, and in exception,
    to the wicked: ``Blessed are all who take refuge in Him.'' (v. 12).
    
% Who they are opposed to: the Lords and His Annointed.
Whereas Psalm 1 describes the ungodly in a general way,
    Psalm 2 identifies the primary characteristic of ungodliness---opposition
    to God and to His Annointed (verse 2).

% They're beef against them (restrictions).
What is the point of conflict that the ungodly have against the Lord?
Verse 3 says,
    \Q{``Let us break their bonds in peices \\
        And cast away their cords from us.''}
That is to say that the sinner's conflict is an attempt to
    overthrow the sovereignty, or rule, of God in his life.
The desire of the wicked is to dethrone God.

This desire climaxed at the cross, 
    as Peter prayed in Acts 4 where he quotes Psalm 2:1-2
    and then applies it to the crucifixion in this way (Acts 4:27-28),
    \Q{ For truly against Your holy Servant Jesus,
        whom You annointed, 
        both Herod and Pontius Pilate, with the Gentiles and the people of Israel,
        were gathered together to do whatever 
        Your hand and Your purpose determined before to be done.}

Psalm 2 then becomes the key to the Christian's understanding of the world,
    and its opposition against God.
As Jesus taught in parable form in Luke 19:14,
    \Q{ But his citizens hated him and sent a delegation after him,
        saying, ``We will not have this man to reign over us.''}

% Reveals the nature of our sin.

\section{The response of the Father}
% contempt
% wrath
% Establishing His Annointed.
\section{The reign of the Son}
% His person,
% His reign
% His victory
\section{The reasoning of the Psalmist}
